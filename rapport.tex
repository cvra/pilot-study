% Déclaration du type de document (report, book, paper, etc...)
\documentclass[a4paper]{paper} 
 
% Package pour avoir Latex en français
\usepackage[utf8]{inputenc}
\usepackage[T1]{fontenc}
 
% Quelques packages utiles
\usepackage{listings} % Pour afficher des listings de programmes
\usepackage{graphicx} % Pour afficher des figures
\usepackage{amsthm}   % Pour créer des théorèmes et des définitions
\usepackage{amsmath}
\usepackage{microtype} % Optical margins FTW
\usepackage{url}
\usepackage{booktabs} % Allows the use of \toprule, \midrule and \bottomrule in tables for horizontal lines
\usepackage{siunitx}
\usepackage{floatrow}
\usepackage{caption}
\usepackage{subcaption}
\usepackage{mhchem}
\usepackage[acronym,smallcaps]{glossaries}


% Début du document
\begin{document}
\begin{titlepage}

\newcommand{\HRule}{\rule{\linewidth}{0.5mm}} % Defines a new command for the horizontal lines, change thickness here

\center % Center everything on the page 
%----------------------------------------------------------------------------------------
%	HEADING 
%----------------------------------------------------------------------------------------
\textsc{\LARGE École Polytechnique Fédérale de~Lausanne}\\[1.5cm] 
\textsc{\Large Méthodes de Production}\\[0.5cm] % Major heading such as course name

%----------------------------------------------------------------------------------------
%	TITLE 
%----------------------------------------------------------------------------------------
\HRule \\[0.4cm]
{ \huge \bfseries Molded Interconnect Devices}\\[0.4cm] % Title of your document
\HRule \\[1.5cm]
 
%----------------------------------------------------------------------------------------
% LOGO EPFL
%----------------------------------------------------------------------------------------
\includegraphics[width=0.8\textwidth]{images/epfl_logo}\\[2cm] 

%----------------------------------------------------------------------------------------
%	AUTHORS 
%----------------------------------------------------------------------------------------

\vfill % Fill the rest of the page with whitespace
\begin{minipage}{0.55\textwidth}
\begin{flushleft} \large
\emph{Étudiants:}\\
Antoine \textsc{Albertelli}\\ 
Quentin \textsc{Herzig}
\end{flushleft}
\end{minipage}
~
\begin{minipage}{0.4\textwidth}
\begin{flushleft} \large
\emph{Enseignants:} \\
Pr. Jacques \textsc{Jacot} \\
Pr. Peter \textsc{Ryser}\\
Dr. Jean-Daniel \textsc{Lüthi}
\end{flushleft}
\end{minipage}\\[2cm]


{\large \today}
\end{titlepage}

% Empty page after title page.
\begin{titlepage}
    ~
\end{titlepage}




\section{General information}
\begin{itemize}
    \item This project study can be published before the contest.
    \item Team name : CVRA (Club Vaudois de Robotique Autonome).
    \item Team location : Renens, Switzerland.
    \item Budget : About 3000 \textsc{chf}, plus sponsors.
\end{itemize}

\section{Robots}
This year the club will engage two robots,
The first one will have two arms, a differential drive and has code name \textit{Debra 5}, since it is the fith evolution of this design.
The small robot hasn't got a name yet.
Both robot share a common architecture, except for the arms, replaced by mission-specific actuators in the small robot.
In this document only points shared by both robots are presented since the design of the mission specific part is not finised yet.

\paragraph{Motors}
Our robots each use two Faulhaber 2232-012-SR motors with a 20:1 reduction, rated at \SI{8.7}{\watt} at \SI{12}{\volt}, overvolted at \SI{14.8}{\volt}.
Those motors allows us to use our max acceleration (wheel slippage) to \SI{0.8}{\meter\per\second}.
Each motor in the robot has an associated motor control board and processor, which will be presented later.

\paragraph{Positioning}
The robots use standard dead reckoning algorithms for positionning itself.
A second pair of wheel, separate from the motor wheels, is used in order to avoid slippage under acceleration or shocks.

\paragraph{Opponent detection}
Our robots have two different avoidance system.
The first one is based on a rotating laser placed on opponent beacons, which allows to have absolute positionning.
The second one is based on a reflex sensor embedded in our robot which detects a reflector on the opponent system.
The redundancy allows for emergency stop in case of failure of the first one or of the pathfinding algorithm.

\begin{figure}[h]
    \begin{center}
        \includegraphics[width=0.6\textwidth]{images/Balise}
        \caption{Fallback beacon system.
            A reflex-type industrial sensor is mounted in the base (blue).
            It emits a LED beam (not laser), which is reflected on the rotating mirror (purple).
            A reflector is placed on the opponent robot, which allows for reliable detection.
            An index sensor (upper left, black) allows the robot to know the relative angle to opponent.
        }
        \label{fig:balise}
    \end{center}
\end{figure}

\paragraph{Power supply}
The robot is powered by a single \SI{14.8}{\volt} lithium-polymer battery, which gives us a run time of at least \SI{40}{\minute}.

\paragraph{Computing}
\begin{figure}[h]
    \begin{center}
        \includegraphics[width=0.6\textwidth]{images/software}
        \caption{Software architecture of our robot.}
        \label{fig:balise}
    \end{center}
\end{figure}

We have three different layers of computing in this year architecture (fig. \ref{fig:software}):
\begin{enumerate}
    \item An x86-based embedded PC running Linux is used for game decisisons and path planning.
    \item The realtime control is done on an ARM board which communicates with the PC using an IP/Ethernet link.
    \item Finally each motor has a motor control board connected to it, which is responsible for speed and torque control, motor protection etc.
        Communication with the main board is done via a CAN bus running at \SI{1}{\mega\hertz}.
\end{enumerate}


This software architecture allows us to easily scale to any number of actuators or sensors in one robot (up to 127 motors per CAN bus/master board).

\section{Team members}
\begin{itemize}
    \item Antoine Albertelli, 21: Software engineer.
    \item Florian Reinhard, 21:  Software engineer.
    \item Patrick Spieler, 22:  Software engineer.
    \item Michael Spieler, 21:  Software engineer.
    \item Pius von Däniken, 23:  Software engineer.
    \item Romain Bersier, 29:  Mechanical engineer, Debra.
    \item Boris Pillionnel, 23:  Mechanical engineer, Nastya.
    \item Guillaume Schaufelberger, 19: Mechanical engineer, Nastya.
    \item Mathieu Rouvinez, 29:  Machining
    \item Jessica Schmid, 20: Electrical engineer.
    \item Dominik Reukauf, 24 : Mechanical Engineer.
\end{itemize}

\clearpage


\section{Sponsors}
\begin{figure}[h!]
    \centering
    \begin{subfigure}[h]{0.15\textheight}
        \includegraphics[width=\textwidth]{images/sponsors/altera}
    \end{subfigure}%
    \hspace{1cm}
    \begin{subfigure}[h]{0.15\textheight}
        \includegraphics[width=\textwidth]{images/sponsors/arrow}
    \end{subfigure}
    \vspace{0.4cm}


\vspace{0.4cm}


    \begin{subfigure}[h]{0.15\textheight}
        \includegraphics[width=\textwidth]{images/sponsors/bossard}
    \end{subfigure}%
    \hspace{1cm}
    \begin{subfigure}[h]{0.15\textheight}
        \includegraphics[width=\textwidth]{images/sponsors/claudePiguet}
    \end{subfigure}
\vspace{0.4cm}


    \begin{subfigure}[h]{0.15\textheight}
        \includegraphics[width=\textwidth]{images/sponsors/faulhaber}
    \end{subfigure}%
    \hspace{1cm}
    \begin{subfigure}[h]{0.15\textheight}
        \includegraphics[width=\textwidth]{images/sponsors/festo}
    \end{subfigure}
\vspace{0.4cm}


    \begin{subfigure}[h]{0.15\textheight}
        \includegraphics[width=\textwidth]{images/sponsors/gardnerDenverThomas}
    \end{subfigure}%
    \hspace{1cm}
    \begin{subfigure}[h] {0.15\textheight}
        \includegraphics[width=\textwidth]{images/sponsors/jauslin}
    \end{subfigure}
\vspace{0.4cm}


    \begin{subfigure}[h]{0.15\textheight}
        \includegraphics[width=\textwidth]{images/sponsors/omron}
    \end{subfigure}%
    \hspace{1cm}
    \begin{subfigure}[h]{0.15\textheight}
        \includegraphics[width=\textwidth]{images/sponsors/phoenixContact}
    \end{subfigure}
\vspace{0.4cm}


    \begin{subfigure}[h]{0.15\textheight}

        \includegraphics[width=\textwidth]{images/sponsors/skf}
    \end{subfigure}%
    \hspace{1cm}
    \begin{subfigure}[h]{0.15\textheight}
        \includegraphics[width=\textwidth]{images/sponsors/posic}
    \end{subfigure}
\vspace{0.4cm}


    \begin{subfigure}[h]{0.15\textheight}
        \includegraphics[width=\textwidth]{images/sponsors/renens}
    \end{subfigure}%
    \hspace{1cm}
    \begin{subfigure}[h]{0.15\textheight}
        \includegraphics[width=\textwidth]{images/sponsors/sick}
    \end{subfigure}
\vspace{0.4cm}


    \begin{subfigure}[h]{0.15\textheight}
        \includegraphics[width=\textwidth]{images/sponsors/pmd}
    \end{subfigure}%
    \hspace{1cm}
    \begin{subfigure}[h]{0.15\textheight}
        \includegraphics[width=\textwidth]{images/sponsors/SwissCNCTechnologies}
    \end{subfigure}
\vspace{0.4cm}


    \begin{subfigure}[h]{0.15\textheight}
        \includegraphics[width=\textwidth]{images/sponsors/SwissMeo}
    \end{subfigure}%
    \hspace{1cm}
    \begin{subfigure}[h]{0.15\textheight}
        \includegraphics[width=\textwidth]{images/sponsors/techniquesLaser}
    \end{subfigure}
\vspace{0.4cm}


\end{figure}


\section{Lasers}
We have two different laser sources in the robot:
\begin{enumerate}
    \item The laser rangefinder (LIDAR).
        This system is certified class 1 by the manufacturer and wasn't modified in any way.
        The datasheet of the system will be provided on contest day.
    \item Main opponent detection system.
        This laser source was taken from a class 2 laser device but our use of the source puts it in class 1.
        This system was certified by the Swiss authorities (METAS) as class 1.
        The certification will be provided on contest day.
\end{enumerate}

\end{document}
